\documentclass{beamer}

\usepackage[utf8]{inputenc}

\usetheme{Madrid}
\usecolortheme{seahorse}
\usefonttheme{structuresmallcapsserif}
\setbeamercovered{dynamic}

\title{Exposée Sensorik}
\author[Y. Höll, G. Muck, C. Pooch, G. N. Tchapda]
	{Yannik~Höll \\ \and Georg~Muck \\ \and Christoph~Pooch \\ \and  Gwladys Noutep Tchapda}
\date{22.04.2021}
\logo{\includegraphics[]{TUBAF-logo.png}}


\beamertemplatenavigationsymbolsempty 

\begin{document}
	\frame{\titlepage}

\AtBeginSection[]{
  \begin{frame}
  \vfill
  \centering
  \begin{beamercolorbox}[sep=8pt,center,shadow=true,rounded=true]{title}
    \usebeamerfont{title}\insertsectionhead\par%
  \end{beamercolorbox}
  \vfill
  \end{frame}
}

\section{Motivation}
\begin{frame}
\frametitle{Motivation}
\begin{itemize}
\item<1-> Ziel: Roboter der sinnvoll über Campus fahren soll
\begin{itemize}
\item<2-4> sinnvolle Navigation
\item<3-4> beachten von Hindernissen wie Menschen oder Schlaglöchern
\item<4> ggf erkennen von Fehlern in anderen Bereichen
\end{itemize}

\item<5-> akkurate \alert<6->{Aufnahme}, \alert<6->{Verarbeitung} und (durch \alert<6->{Verarbeitung}) sinnvolle \alert<6->{Bereitstellung} der Sensordaten
\end{itemize}
\end{frame}


\section{Organisation \& Ablauf}
\begin{frame}
\frametitle{Organisation \& Ablauf}
\begin{itemize}
\item<1-> \alert{Meeting} am Anfang und am Ende der "Arbeitswoche"
\item<2-> Aufgaben zu geregelten Zeiten erledigen
\item<3-> feste \alert{Verbindlichkeiten}
\item<4-> Kommunikation via \alert{Discord} und Datenaustausch via \alert{GitHub}
\end{itemize}
\end{frame}

\begin{frame}
\frametitle{Organisation \& Ablauf}
\begin{itemize}
\item<1-> Vorbereitung
\begin{itemize}
\item<2-3> Definieren des Problems
\item<3> Kommunikation mit anderen Gruppen
\end{itemize}
\item<4-> Recherche \& Planung
\begin{itemize}
\item<5-8> Recherchieren der benötigten Mittel
\item<6-8> Ablaufpläne \& UML-Klassendiagramm erstellen
\item<7-8> Analysieren der GitHub-Bibliotheken
\item<8> Ansätze/Ideen der Integration
\end{itemize}
 \item<9-> Umsetzung
\begin{itemize}
\item<10-12> Schreiben der einzelnen Nodes \& Eigenschaften
\item<11-12> grobe Implementierung aller Sensoren
\item<12> Verfeinerung
\end{itemize}
\end{itemize}
\end{frame}

\end{document}